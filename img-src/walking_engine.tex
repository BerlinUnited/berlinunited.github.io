\documentclass{standalone}% For the example only, any class will do

\usepackage{tikz}
\usetikzlibrary{positioning}% To get more advances positioning options
\usetikzlibrary{arrows.meta}
\usetikzlibrary{fit}
\usetikzlibrary{calc}
\usetikzlibrary{backgrounds}


\begin{document}
	\tikzset{%
		>={Latex[width=2mm,length=2mm]},
		% Specifications for style of nodes:
		base/.style = {rectangle, rounded corners, draw=black,
			minimum width=4cm, minimum height=1cm,
			text centered, font=\sffamily},
		module/.style = {base, minimum width=2.5cm, fill=green!15, font=\ttfamily},
		sub_module/.style = {base, minimum width=2.5cm, fill=white!15, font=\ttfamily},
		rep/.style    = {base, minimum width=2.5cm, fill=blue!15, font=\ttfamily},
		robot/.style  = {base, minimum width=2.5cm, fill=gray!15, font=\ttfamily},
	}	
	\begin{tikzpicture}[node distance=1.5cm, every node/.style={fill=white, font=\sffamily}, align=center]
		\node (Cognition)                  [sub_module, minimum width=8cm] {Cognition};
		\node (inv2) 		       [inner sep=0,minimum size=0,below= of Cognition]{}; % invisible node
		\node (MotionRequest)              [rep, left = of inv2]  {Motion Request};
		\node (MotionStatus)               [rep, right = of inv2]  {Motion Status};
		\node (FootStepPlanner)            [sub_module, below = of MotionRequest] {Foot Step Planner};
		\node (StepBuffer)                 [rep, right = of FootStepPlanner]  {Step Buffer};
		\node (inv1) 		       [inner sep=0,minimum size=0,right= 0cm and 2cm of StepBuffer]{}; % invisible node
		\node (ZMPPlanner)                 [sub_module, below = of StepBuffer] {ZMPPlanner};
		\node (ZMPReferenceBuffer)         [rep, below = of ZMPPlanner]  {ZMP Reference Buffer};
		\node (ZMPPreviewController)       [sub_module, below = of ZMPReferenceBuffer] {ZMP Preview Controller};
		\node (FootTrajectoryGenerator)    [sub_module, right = of inv1] {Foot Trajectory Generator};
		\node (HipRotationOffsetModifier)  [sub_module, below = of FootTrajectoryGenerator] {Hip Rotation Offset Modifier};
		\node (LiftingFootCompensator)     [sub_module, below = of HipRotationOffsetModifier] {Lifting Foot Compensator};
		\node (TargetCoMFeetPose)          [rep, below = of ZMPPreviewController]  {Target CoM Feet Pose};
		\node (IKOpt)                      [sub_module, below = of TargetCoMFeetPose] {IK Optimizer Hip};
		\node (TargetHipFeetPose)          [rep, below = of IKOpt]  {Target Hip Feet Pose};
		\node (TorsoRotationStabilizer)    [sub_module, right = of TargetHipFeetPose] {Torso Rotation\\Stabilizer};
		\node (IK)                         [sub_module, below = of TargetHipFeetPose] {Inverse Kinematics};
		\node (MotorJoints)                [rep, below = of IK]  {Motor Joints};
		\path let \p1=(IK), \p2=(TorsoRotationStabilizer) in node[sub_module] (FeetStabilizer) at (\x2,\y1) {FeetStabilizer};
		\node (CoMErrors)                  [rep, left = of FootStepPlanner]  {CoM Errors};
		\node (CoMErrorProvider)           [sub_module, below = of CoMErrors] {CoM Error Provider};
		\node (CommandPoseBuffer)          [rep, left = of TargetCoMFeetPose]  {CommandPoseBuffer};
		\path let \p1=(MotorJoints), \p2=(CoMErrorProvider) in node[rep] (KinematicChainSensor) at (\x2,\y1) {Kinematic Chain Sensor};
		\node (AccelerometerData)          [rep, below right = of MotorJoints]  {Accelerometer\\Data};
		\node (IMUModel)                   [module, right = of AccelerometerData]  {IMU-Model};
		\node (GyrometerData)              [rep, right = of IMUModel]  {Gyrometer\\Data};
		\path let \p1=(MotorJoints), \p2=(IMUModel) in node[rep] (IMUDataInertialModel) at (\x2,\y1) {IMUData / \\InertialModel};
		\node (Robot)                      [robot, below  = 3.5cm and 0cm of MotorJoints]  {Robot};
		
		\coordinate [right = of HipRotationOffsetModifier] (join1);
		\path (TorsoRotationStabilizer) -- (FeetStabilizer) node[shape=coordinate, midway] (join2) {};
		
		\begin{scope}[on background layer]
			\node (Motion)	[sub_module, inner sep=0.5cm, fit=(GyrometerData) (MotionStatus) (MotionRequest) (KinematicChainSensor)] {};
			\node (Walk2018)	[module, inner sep=0.8cm, fit=(join1) (IK) (CoMErrorProvider) (CoMErrors)] {};
			\node [below right, opacity=0, text opacity=1] at (Walk2018.north west) {Walk2018-Module};
			\node [below right, opacity=0, text opacity=1] at (Motion.north west) {Motion};
		\end{scope}
		
		\draw[->] (MotionRequest.north|-Cognition.south) -- (MotionRequest);
		\draw[->] (MotionStatus) -- (MotionStatus.north|-Cognition.south);
		\draw[->] (MotionRequest) -- (FootStepPlanner);
		\draw[->] (FootStepPlanner) -- (StepBuffer);
		\draw[->] (StepBuffer) -- (ZMPPlanner);
		\draw[->] (StepBuffer) -- (FootTrajectoryGenerator);
		\draw[->] (inv1) |- (HipRotationOffsetModifier);
		\draw[->] (inv1) |- (LiftingFootCompensator);
		\draw[->] (ZMPPlanner) -- (ZMPReferenceBuffer);
		\draw[->] (ZMPReferenceBuffer) -- (ZMPPreviewController);
		\draw[->] (ZMPPreviewController) -- (TargetCoMFeetPose);
		\draw (FootTrajectoryGenerator)   -| (join1);
		\draw (HipRotationOffsetModifier) -- (join1);
		\draw (LiftingFootCompensator)    -| (join1);
		\draw[->] (join1) |- (TargetCoMFeetPose);
		\draw[->] (TargetCoMFeetPose) -- (IKOpt);
		\draw[->] (IKOpt) -- (TargetHipFeetPose);
		\draw[->] (TargetHipFeetPose) -- (IK);
		\draw[->] (FeetStabilizer) |- (MotorJoints);
		\draw[->] (IK) -- (MotorJoints);
		\draw[->] (MotorJoints) -- (Robot);
		\draw (IMUDataInertialModel) |- (join2);
		\draw[->] (join2) -- (TorsoRotationStabilizer);
		\draw[->] (join2) -- (FeetStabilizer);
		\draw (GyrometerData) |- (join2);
		\draw[->] (TorsoRotationStabilizer) -- (TargetHipFeetPose);
		\draw[->] (CoMErrors) -- (FootStepPlanner);
		\draw[->] (CoMErrorProvider) -- (CoMErrors);
		\draw[->] (Robot) -| (AccelerometerData);
		\draw[->] (Robot) -| (KinematicChainSensor);
		\draw[->] (Robot) -| (GyrometerData);
		\draw[->] (KinematicChainSensor) -- (CoMErrorProvider);
		\draw[->] (CommandPoseBuffer) |- (CoMErrorProvider);
		\draw[->] (TargetCoMFeetPose) -- (CommandPoseBuffer);
		\draw[->] (IMUModel) -- (IMUDataInertialModel);
		\draw[->] (AccelerometerData) -- (IMUModel);
		\draw[->] (GyrometerData) -- (IMUModel);
	\end{tikzpicture}
\end{document}